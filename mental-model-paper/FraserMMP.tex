\documentclass[11pt, oneside]{article}
\usepackage{hyperref}
\begin{document}

\title{The Second Screen}
\author{Michael Fraser}
\maketitle

\begin{abstract}
For this topic, 
you are asked to investigate the history of second 
screen technologies and to place the latest such 
offerings in the context of this history
\end{abstract}

\tableofcontents

\section{Introduction}
Almost everyone these days has a tablet or smart phone device. As the various forms of media begin to recognize this face, they begin to utilize it to their advantage. It used to be that people would sit down and watch TV with whatever TV screen they had at the time, and interaction with their favorite TV shows was limited to this action. As technology advanced, the common TV watcher was granted easy accessibility to a second screen of sorts: a tablet or smart phone. 

The same is true for users of entertainment systems and computers; multiple screens are available to push the limits of interaction with common devices. Applications and devices are now able to allow for a more in-depth interaction with the systems people have interacted with in a limited way for so long. Are these so called "second screens" going to replace the traditional forms of interaction with common systems, or will it remain a specialized tool for specific applications?

%Second (and more) monitors have been available for decades, yet recently they seem to have come into their own - so called “second screen” applications and devices have taken what used to represent “just more space” and laid claim to new families of applications whose very existence owes itself to having that “second screen.” The main question: Is the second screen poised to become a new baseline technology, or does it remain a niche for specialized or vertical applications?

\section{Background}
A simple Google search reveals that this is not a new concept. Since the development of social networking sites like Facebook and Twiter, the media has been engaging their supporters through mass messages. Prime time shows like Covert Affairs post statuses on the show's facebook page engaging followers about the decisions made in the show, whereas shows like Tosh.0 have the host direct the viewers into tweeting about things Tosh has said. 

Studies completed by Twitter have found that when a show is airing that directly integrates tweets into the content, there is a major increase in the number of tweets engaging the show's hashtag subjects \cite{TwitterTV}. Chris Gorham of Covert Affairs was interviewed by Mashable on the effects of the second screen and the show's success. He believes Twitter is a way of sharing the behind-the-scenes process with his fans \cite{MashableChris}.



\section{Methods}
Methods

\section{Discussion}
Whether it is a facebook status asking followers about a certain moment of a show or the host of a show telling viewers to tweet what they think, social networking is a big part of the second screen movement. 

\section{Conclusions}
Conclusions

\bibliography{mental.bib}{}
\bibliographystyle{plain}

\end{document}
