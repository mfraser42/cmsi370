\documentclass[11pt, oneside]{article}
\usepackage{graphicx}
\usepackage{url}
\begin{document}

\title{A Text-Editor "Dream" Interface}
\author{Michael Fraser}
\maketitle

\tableofcontents

\section{Background}
Common text editors have many of the same features, like language-specific keyword highlighting, automatic text indentation, and easy customization. These editors often use a system of menus and forms to allow the user to modify preferences, manage the file, or access additional tools. As with most applications, the text editors also allow the user to directly manipulate the shape of a session window, the location of the session window, and the positioning of session tabs. Most of the actual text writing is done by the typing of a keyboard, with no alternative forms of text input. Text editors are very commonly used by programmers, as they offer wide ranges of file types and programs do not require the unnecessary font editing capabilities of a word processor.

As all these text editors have many of the same features, they all maintain similar levels of efficiency, learnability, rememberability, errors, and satisfaction. The menus and forms that allow the users to easily find a desired command or tool allow for easy learnability of the editor. It also eliminates some need for high rememberability, as the forms and menus allow for commands to be quickly found again. The editors achieve similar levels of efficiency with the forms and menus, requiring several steps to reach each command. They also allow more experienced users to utilize increased efficiency with keyboard shortcuts. Satisfaction levels for the editors rely on the features of the editors and the ability for users to edit their preferences, and vary from editor to editor. %errors!?!?

Many aspects and features of today's common text editors would be included in a the design of an ideal text editor, but current text editors lack several features that could bring text editing to a new level. As new technologies are developed, new methods of interaction are developed that could alter the way text editors allow for text to be input to a document. Text editors could even take some features of common word processors to make their use as programming tools more effective and satisfying. 

\section{}


\end{document}

%\bibliography{mental.bib}{}
%\bibliographystyle{plain}

% theories, principles and guidelines
%7 stage sof action, oAI, usability metrics,
%fitt's law, tagazinni's 16, etc
% interaction styles: direct manipulation, menus forms and dialogues, command-line, natural language
% everything together: dream interface, interface that would be as good as possible
% lego exercise: affordance: objects and amterials send you signals, tell you things that help you understand what those things are for
% when it all comes together, the user should get the same picture you had
% 

